\documentclass[a4paper,12pt]{article}
\usepackage{Report}

%%%%%%%%%%%%%%%%%%%%%%%%%%%%%%%%%%%%%%%%
% 实验信息
%%%%%%%%%%%%%%%%%%%%%%%%%%%%%%%%%%%%%%%%
\newcommand{\theTitle}{\ \ \ \ \ \ \ \ \ \ \ \ xxxx\ \ \ \ \ \ \ \ \ \ \ \ \ \ \ \ }
\newcommand{\theAuthorName}{\ \ \ \ xxx\ \ \ \ \ \ }
\newcommand{\theAuthorID}{xxxxxxxxxxxx}
\newcommand{\theType}{xxxxx}
\newcommand{\theGrade}{xxxx级}
\newcommand{\theMajor}{xxxx}
\newcommand{\theDepartment}{xxx学院}
\newcommand{\theTutor}{xxx}
\newcommand{\theTitleOfTutor}{xx}
\newcommand{\theLab}{xxx}
\newcommand{\theDate}{xxxx年x月xx日}


\begin{document}
%%%%%%%%%%%%%%%%%%%%%%%%%%%%%%%%%%%%%%%%
% 封面
%%%%%%%%%%%%%%%%%%%%%%%%%%%%%%%%%%%%%%%%
\begin{titlepage}
\vspace*{8\baselineskip} 

\begin{center}
    \songti\fontsize{30.0pt}{50.0pt}
    \textbf{本\quad 科\quad 实\quad 验\quad 报\quad 告}
\end{center}

\vspace*{8\baselineskip} 

\begin{table}[H]
\centering
\zihao{3}
\begin{tabular}{ccc}
    实验名称: &  & \theTitle \\ \cline{3-3}
\end{tabular}
\end{table}

\vspace*{3\baselineskip} 

\begin{table}[H]
\renewcommand\arraystretch{1.5} 
\centering
\zihao{4}
\begin{tabular}{cccccc}
学\qquad 员: &  & \theAuthorName & \ \ \ 学\qquad 号: &  & \theAuthorID \\ \cline{3-3} \cline{6-6} 
培养类型: &  & \theType & \ \ \ 年\qquad 级: &  & \theGrade \\  \cline{3-3} \cline{6-6} 
专\qquad 业: &  & \theMajor & \ \ \ 所属学院: &  & \theDepartment \\  \cline{3-3} \cline{6-6} 
指导教员: &  & \theTutor & \ \ \ 职\qquad 称: &  & \theTitleOfTutor \\  \cline{3-3} \cline{6-6} 
实\; 验\; 室: &  & \theLab & \ \ \ 实验日期: &  & \theDate \\ \cline{3-3} \cline{6-6} 
\end{tabular}
\end{table}

\end{titlepage}

%%%%%%%%%%%%%%%%%%%%%%%%%%%%%%%%%%%%%%%%
% 填写说明
%%%%%%%%%%%%%%%%%%%%%%%%%%%%%%%%%%%%%%%%
\begin{center}
    \heiti\zihao{4}
    《实验报告》填写说明
\end{center}

\vspace*{1\baselineskip} 

\begin{enumerate}
\songti\zihao{-4}
\item 学员完成人才培养方案和课程标准所要求的每个实验后,均须提交实验报告。
\item 实验报告无需打印,电子版上传EDUCODER。
\item 实验报告内容编排及打印应符合以下要求:
    \begin{enumerate}
    \fangsong\zihao{-4}
    \item 上下左右各侧的页边距均为3cm;缺省文档网格:字号为小4号,中文为宋体,英文和阿拉伯数字为Times New Roman,每页30行,每行36字;页脚距边界为2.5cm,页码置于页脚、居中,采用小5号阿拉伯数字从1开始连续编排,封面不编页码。
    \item 报告正文最多可设四级标题,字体均为黑体,第一级标题字号为4号,其余各级标题为小4号;标题序号第一级用“一、”“二、”……,第二级用“(一)”“(二)” ……,第三级用“1.”“2.” ……,第四级用“(1)”“(2)” ……,分别按序连续编排。
    \item 正文插图、表格中的文字字号均为5号。
    \end{enumerate}
\end{enumerate}

\newpage

%%%%%%%%%%%%%%%%%%%%%%%%%%%%%%%%%%%%%%%%
% 正文
%%%%%%%%%%%%%%%%%%%%%%%%%%%%%%%%%%%%%%%%
\section{实验目的和要求}
This is a test document with English and Chinese text. 这是一段测试文本,包含英文和中文。This is a test document with English and Chinese text. 这是一段测试文本,包含英文和中文。This is a test document with English and Chinese text. 这是一段测试文本,包含英文和中文。

This is a test document with English and Chinese text. 这是一段测试文本,包含英文和中文。This is a test document with English and Chinese text. 这是一段测试文本,包含英文和中文。This is a test document with English and Chinese text. 这是一段测试文本,包含英文和中文。This is a test document with English and Chinese text. 这是一段测试文本,包含英文和中文。

\section{实验内容和原理}
\subsection{实验原理}
This is a test document with English and Chinese text. 这是一段测试文本,包含英文和中文。
\subsection{实验内容}



\section{操作方法与实验步骤}


\section{实验结果与分析}


\section{实验思考}


\end{document}
